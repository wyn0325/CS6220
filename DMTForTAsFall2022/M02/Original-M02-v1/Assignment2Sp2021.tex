%%%%%%%%%%%%%%%%%%%%%%%%%%%%%%%%%%%%%%%%%%%%%%%%
% 1. Document class 
\documentclass[letter-paper,12pt]{article} % This defines the style of your paper
%%%%%%%%%%%%%%%%%%%%%%%%%%%%%%%%%%%%%%%%%%%%%%%%
% 2. Packages
\usepackage[top = 2.5cm, bottom = 2.5cm, left = 2.5cm, right = 2.5cm]{geometry} 
\usepackage[T1]{fontenc}
\usepackage[utf8]{inputenc}
\usepackage{hyperref}
\hypersetup{
    colorlinks=true,
    urlcolor={blue},
}

\usepackage{multirow} % Multirow is for tables with multiple rows within one cell.
\usepackage{booktabs} % For even nicer tables.
\usepackage{graphicx} 
\usepackage{setspace}
\setlength{\parindent}{0in}
\usepackage{float}
\usepackage{fancyhdr}
\usepackage{titlesec}
\usepackage{url}
\usepackage{amsmath,amssymb,amsthm,bm}
\usepackage{subfigure}
\usepackage{subcaption}

\titleformat*{\section}{\large\bfseries}
\titleformat*{\subsection}{\bfseries}
%%%%%%%%%%%%%%%%%%%%%%%%%%%%%%%%%%%%%%%%%%%%%%%%
% 3. Header (and Footer)
\pagestyle{fancy} % With this command we can customize the header style.
\fancyhf{} % This makes sure we do not have other information in our header or footer.

% \title{CS 6220 Data Mining --- Assignment 2}
% \date{}

\begin{document}
% \thispagestyle{empty} % This command disables the header on the first page. 
% \maketitle

% \hline
\begin{center}
\begin{Large}
CS 6220 Data Mining --- Assignment 2
\end{Large}
\end{center}

\hline
\hline
~\\~\\~\\

\begin{center}
\begin{Large}
\textbf{Exploring Data with Pandas}
\end{Large}
\end{center}
~\\~\\

Prior to beginning your work on this assignment, download and run this \href{https://goo.gl/BFprVd}{notebook file} (\url{https://goo.gl/BFprVd}), which will cover some basics on data exploration, loading data, extracting basic statistics from various features, and generating visualizations.\\

\textbf{Assignment Description:}\\
This assignment will require that you implement and interpret some of the data understanding concepts that were introduced in class, such as summary statistics and data visualization. Further, you will be working with real-world data retrieved from an online repository, and while you will be asked to utilize a variety of modules and functions, these have all been covered in the notebook files that were shared. Keep in mind that the main objective of this assignment is to highlight the insights that we can derive from the data understanding process – the coding aspect is secondary. Accordingly, you are welcome to consult any online documentation and/or code so long as all references and sources are properly cited. You are also encouraged to use code libraries, but be sure to acknowledge any source code that was not written by you by mentioning the original author(s) directly in your source code (comment or header).\\


\textbf{Submission:}\\
Submit your ipynb file through the Assignment Submission Portal as done in Assignment 1.

\section{Iris Dataset [60 Points]}
Using your own module of choice (we recommend pandas), download the Iris flower dataset available at (\url{http://archive.ics.uci.edu/ml/machine-learning-databases/iris/iris.data}) into a Data-Frame. For more details about the dataset and to obtain the feature names, check (\url{http://archive.ics.uci.edu/ml/datasets/Iris}). It is always recommended that you familiarize yourself with the data you intend to use for data mining purposes. The Iris dataset, in particular, has a rich history, having been introduced in 1936 by Sir Ronald Fisher, often considered one of the fathers of modern statistical theory.

\subsection{Summary Statistics [10 Points]}
Compute and display summary statistics for each feature available in the dataset. These must include the minimum value, maximum value, mean, range, standard deviation, variance, count, and 25:50:75\% percentiles.

\subsection{Data Visualization [25 Points]}
\textbf{Histograms:} To illustrate the feature distributions, create a histogram for each feature in the dataset. You may plot each histogram individually or combine them all into a single plot. When generating histograms for this assignment, use the default number of bins. Recall that a histogram provides a graphical representation of the distribution of the data.\\

\textbf{Box Plots:} To further understand the data, create a boxplot for each feature in the dataset.  Present all the boxplots into a single plot. Recall that a boxplot provides a graphical representation of the location and variation of the data through their quartiles; they are especially useful for comparing distributions and identifying outliers.\\

\textbf{Pairwise Plot:} To understand the relationship between the features, create a scatter plot for each pair of the features. If there are are $n$ features in the dataset, there should be $nC2$ plots.\\

\textbf{Class-wise Visualization:} Create histograms for each feature in a similar way for each of the different classes present in the data. \\

\subsection{Conceptual Questions [25 Points]}
Answer the following questions about the analysis you just performed. Include the answers to these questions as text content (using markdown or text cells on Jupyter notebook) in the same notebook file used for visualization.

\begin{enumerate}
    \item How many features are there? What are the types of the features (e.g., numeric, nominal, discrete, continuous)?
    
    \item From the histograms of the whole data, how do the shapes of the histograms for petal length and petal width differ from those for sepal length and sepal width? Is there a particular value of petal length (which ranges from 1.0 to 6.9) where the distribution of petal lengths (as illustrated by the histogram) could be best segmented into two parts?
    
    \item Based upon these boxplots, is there a pair of features that appear to have significantly different medians? Recall that the degree of overlap between variability is an important initial indicator of the likelihood that differences in means or medians are meaningful. Also, based solely upon the box plots, which feature appears to explain the greatest amount of the data?
    
    \item From the pairwise plots of the features, which features are most correlated from the plots? Mention at least three pairs. 
    
    \item Compare the histograms of each class to the histograms of the whole dataset. What differences do you see in the shapes? 
    
\end{enumerate}


\section{Air Quality Dataset [40 Points]}
Download the Air Quality dataset from  (\url{http://archive.ics.uci.edu/ml/machine-learning-databases/00360/AirQualityUCI.zip}). Note that this dataset is much larger than
the Iris dataset, both with respect to the number of instances and the number of features. A description of this dataset can be found at (\url{http://archive.ics.uci.edu/ml/datasets/Air+Quality}).

Download the dataset in your machine, and then unzip. Use the \textit{AirQualityUCI.xlsx} file for the data. You can use \texttt{pandas.read\_excel(`AirQualityUCI.xlsx')} to read the file in DataFrame. 

\subsection{Summary Statistics [5 Points]}
As in Section 1, Compute and display summary statistics for each feature available in the dataset. These must include the minimum value, maximum value, mean, range, standard deviation, variance, count, and 25:50:75\% percentiles.

\subsection{Data Visualization [15 Points]}
 Also as in Section 1, create histograms and boxplots for the dataset with and without outliers. You may use \texttt{showfliers=False} to remove outliers from the boxplots. 

\subsection{Conceptual Questions [20 Points]}
Answer the following questions about the analysis you just performed. Include the answers to this questions as text content (using markdown or text cells on Jupyter notebook) in the same notebook file used for visualization.
\begin{enumerate}
    \item From the histograms, what abnormality can you see? 
    \item What abnormality can you see from the summary statistics?
    \item How can you remove the abnormality from the data? 
    \item Show how the histograms look after removing the abnormalities from the data?
    
\end{enumerate}



\end{document}